\documentclass[12pt,fleqn]{article}

\usepackage[utf8]{inputenc}
\usepackage[T2A]{fontenc}
\usepackage{amssymb,amsmath,mathrsfs,amsthm}
\usepackage[russian]{babel}
\usepackage{graphicx}
\usepackage[footnotesize]{caption2}
\usepackage{indentfirst}
%\usepackage[ruled,section]{algorithm}
%\usepackage[noend]{algorithmic}
%\usepackage[all]{xy}

% Параметры страницы
\textheight=24cm
\textwidth=16cm
\oddsidemargin=5mm
\evensidemargin=-5mm
\marginparwidth=36pt
\topmargin=-1cm
\footnotesep=3ex
%\flushbottom
\raggedbottom
\tolerance 3000
% подавить эффект "висячих стpок"
\clubpenalty=10000
\widowpenalty=10000
\renewcommand{\baselinestretch}{1.1}
\renewcommand{\baselinestretch}{1.5} %для печати с большим интервалом

\begin{document}

\begin{titlepage}
\begin{center}
    Московский государственный университет имени М. В. Ломоносова

    \bigskip
    \includegraphics[width=50mm]{msu.pdf}

    \bigskip
    Факультет Вычислительной Математики и Кибернетики\\
    Кафедра Автоматизации Систем Вычислительных Комплексов\\[10mm]

    \textsf{\large\bfseries
        ДИПЛОМНАЯ РАБОТА СТУДЕНТА 420 ГРУППЫ\\[10mm]
        Расчёт освещённости при помощи метода излучательности на графических процессорах для интерактивных приложений
    }\\[10mm]

    \begin{flushright}
        \parbox{0.5\textwidth}{
            Выполнил:\\
            студент 4 курса 420 группы\\
            \emph{Щербаков Александр Станиславович}\\[5mm]
            Научный руководитель:\\
            к.ф-м.н.\\
            \emph{Фролов Владимир Александрович}
        }
    \end{flushright}

    \begin{tabular}{p{0.45\textwidth}p{0.45\textwidth}}
        И. о. заведующего кафедрой\newline
        Автоматизации Систем\newline
        Вычислительных Комплексов
        &
        ~\newline~\newline
        \hfill\hbox to 0.45\textwidth{\hrulefill~В. К. Власов}
    \\[20mm]
        К защите допускаю\newline
        \hbox to 0.4\textwidth{<<\hbox to 12mm{\hrulefill}>> \hrulefill~2017 г.}
        &
        К защите рекомендую\newline
        \hbox to 0.45\textwidth{<<\hbox to 12mm{\hrulefill}>> \hrulefill~2017 г.}
    \end{tabular}

    \vspace{\fill}
    Москва, 2017
\end{center}
\end{titlepage}

\newpage
\renewcommand{\contentsname}{Содержание}
\tableofcontents

\newpage
\begin{abstract}
    Данный документ является образцом оформления дипломной работы для студентов кафедры 
    Математических методов прогнозирования ВМК~МГУ. 
    Приведённые ниже рекомендации взяты из~статьи
    <<Написание отчётов и статей (рекомендации)>>
    на~вики"~ресурсе \texttt{www.MachineLearning.ru}.
    Студенты, готовящие дипломную работу к~защите, 
    могут найти много полезной информации также в~статьях 
    <<Научно-исследовательская работа (рекомендации)>>,
    <<Подготовка презентаций (рекомендации)>>,
    <<Защита выпускной квалификационной работы (рекомендации)>>
    на~том~же ресурсе. 

    Аннотация обычно содержит 
    краткое описание постановки задачи и~полученных результатов,
    одним абзацем на 10--15 строк.
    Цель аннотации "--- обозначить в~общих чертах, о~чём работа,
    чтобы человек, совершенно не~знакомый с~данной работой,
    понял, интересна~ли ему эта тема, и~стоит~ли читать дальше.
    Аннотация собирается в~последнюю очередь
    путем легкой модификации наиболее важных и~удачных фраз из введения и~заключения.
\end{abstract}

\newpage
\section{Введение}

\section{Постановка задачи}

\subsection{Неформальная постановка задачи}

Требуется уменьшить вычислительные затраты при расчёте вторичного освещения в~3D-сцене с~динамическими источниками света методом излучательности.

\subsection{Формальная постановка задачи}

Для заданной трёхмерной сцены (геометрия, источники света, материалы) первичным освещением называется освещение исходящее непосредственно из~источников света. Вторичным освещением называется освещение, полученное путём многократного отражения первичного освещения от~поверхностей сцены.

Необходимо, используя данные о~геометрии и~материалах сцены, провести её предобработку. Полученные данные используются во~время отрисовки сцены в~интерактивном режиме с~изменением конфигурации освещения.

\subsection{Обзор литературы}

Лучше, чтобы название этого подраздела было содержательным, 
например, общепринятым названием задачи, проблемы или метода,
рассматриваемого в~данной работе. 

Перечисляются подходы, методы, факты, на которые существенно опирается данная работа, 
но~которые могут быть не~известны широкому кругу читателей.
Здесь ссылки на литературу обязательны. 
Теоремы только формулируются, но не~доказываются.

Данный раздел преследует две цели. 
Во-первых, сделать работу самодостаточной~--- дать необходимый минимум информации тем читателям,
которые не~очень хорошо ориентируются в~теме, но желают поближе познакомиться именно с~данной работой.
Во-вторых, облегчить сопоставление полученных автором результатов с~ранее известными.

\section{Новые подходы и~результаты}

Название этого раздела обязательно надо заменить на~содержательное. 
В~этом разделе, как правило, много подразделов. 

В~дипломной работе не~стоит делать более двух уровней,
достаточно разделов и~подразделов.
Будете писать диссертацию или монографию~--- сделаете три уровня. 
  
\section{Вычислительные эксперименты}

Цель данного раздела:
продемонстрировать, что предложенная теория работает на практике;
показать границы её применимости;
рассказать о~новых экспериментальных фактах.

Чисто теоретические работы могут вообще не~содержать раздела экспериментов
(не~работает, ну и~не~надо~--- зато теория красивая).
Кстати, теоретики имеют право не~догадываться, где, кому и~когда их теории пригодятся.

\subsection{Исходные данные и~условия эксперимента}
Описывается прикладная задача, параметры анализируемых данных 
(например, сколько объектов, сколько признаков, каких они типов), 
параметры эксперимента 
(например, как производился скользящий контроль). 

\subsection{Результаты эксперимента}
Результаты экспериментов представляются в~виде таблиц и~графиков. 
Объясняется точный смысл всех обозначений на графиках, строк и~столбцов в~таблицах. 

\subsection{Обсуждение и~выводы}
Приводятся выводы: 
в~какой степени результаты экспериментов согласуются с~теорией? 
Достигнут ли желаемый результат? 
Обнаружены ли какие-либо факты, не~нашедшие объяснения, и~которые нельзя списать на «грязный» эксперимент?

Обсуждаются основные отличия предложенных методов от известных ранее. 
В~чем их преимущества? 
Каковы границы их применимости? 
Какие проблемы удалось решить, а~какие остались открытыми? 
Какие возникли новые постановки задач?

\section{Заключение}

В~квалификационных работах последний раздел нужен для того, чтобы 
конспективно перечислить основные результаты, полученные лично автором. 

Результатами, в~частности, являются:
\begin{itemize}
\item 
    Предложен новый подход к\dots
\item 
    Разработан новый метод\dots, позволяющий\dots
\item 
    Доказан ряд теорем, подтверждающих (опровергающих), что\dots
\item 
    Проведены вычислительные эксперименты\dots,
    которые подтвердили / опровергли / привели к~новым постановкам задач.
\end{itemize}
    
Цель данного раздела: доказать квалификацию автора. 
Даже беглого взгляда на заключение должно быть достаточно, чтобы стало ясно: 
автору удалось решить актуальную, трудную, ранее не~решённую задачу, 
предложенные автором решения обоснованы и~проверены.

Иногда в~Заключении приводится список направлений дальнейших исследований.

\newpage
Список литературы необходим в~любой научной публикации. 
В дипломной работе он~обязателен. 
Дурным тоном считается:
ссылаться на работы только одного-двух авторов (например, себя или шефа);
ссылаться на слишком малое число работ;
ссылаться только на очень старые работы;
ссылаться на работы, которых автор ни разу не видел;
ссылаться на~работы, которые не~упоминаются в~тексте
или которые не~имеют отношения к~данному тексту.

\renewcommand{\bibname}{Список литературы}
\addcontentsline{toc}{section}{\bibname}

\nocite{hastie09elements,bishop06pattern,zhuravlev06recognition,zhuravlev78prob33,shlezinger04ten,boucheron05theory}

\def\BibUrl#1.{}\def\BibAnnote#1.{}
%\def\BibUrl#1{\\{\footnotesize\tt\def~{\char126} http://#1}}
\bibliographystyle{gost71s}
\bibliography{MachLearn}

\end{document}
