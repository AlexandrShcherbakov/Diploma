\documentclass[unicode, dvipsnames]{beamer}

\usetheme{Pittsburgh} %тема
\usecolortheme{seahorse} %цветовая гамма
%Выбирать тему и цветовую гамму презентации очень удобно на http://www.hartwork.org/beamer-theme-matrix/

\usefonttheme[onlylarge]{structurebold} % названия и текст в колонтитулах выводится полужирным шрифтом.
\usefonttheme[onlymath]{serif}  % привычный шрифт для математических формул
\setbeamerfont*{frametitle}{size=\normalsize,series=\bfseries} % шрифт заголовков слайдов

%Русификация
\usepackage[utf8]{inputenc}
\usepackage[russian]{babel}

%Для вставки рисунков
\usepackage{graphics, graphicx}
\usepackage{movie15}
\usepackage{epstopdf}

\usepackage[nopar]{lipsum} %для генерации большого текста

%Для вывода листинга TeXовского кода
\usepackage{listings}
\usepackage{color}
\usepackage{textcomp}

\usepackage{array}
\newcolumntype{L}[1]{>{\raggedright\let\newline\\\arraybackslash\hspace{0pt}}m{#1}}
\newcolumntype{C}[1]{>{\centering\let\newline\\\arraybackslash\hspace{0pt}}m{#1}}
\newcolumntype{R}[1]{>{\raggedleft\let\newline\\\arraybackslash\hspace{0pt}}m{#1}}

\addtobeamertemplate{navigation symbols}{}{ \hspace{1em} \raisebox{2pt}{\usebeamerfont{footline}%
    \insertframenumber / \inserttotalframenumber}}
\setbeamercolor{navigation symbols}{fg=MidnightBlue}

\title{Расчёт освещённости при помощи метода излучательности на графических процессорах для интерактивных приложений}
\author{\small Щербаков Александр \\ \vspace{10pt} Научный руководитель: \\ к.ф.-м.н. Фролов В. А.}
%\institute[ВМК МГУ]{ВМК МГУ}
\date{\small 3 июня 2017}

\logo{\includegraphics[height=1cm]{logo.png}\vspace{206pt}}
\addtobeamertemplate{title page}{\small \begin{center}Московский государственный университет имени М.В.~Ломоносова\\
Факультет Вычислительной Математики и Кибернетики \\
Кафедра Автоматизации Систем Вычислительных Комплексов \\
Лаборатория компьютерной графики и мультимедиа \end{center} \vspace{-10pt}}{}

%%Перед началом каждого раздела выводится его название в рамках общего содержания
%\AtBeginSection[]
%{
%  \begin{frame}{Содержание презентации}
%    \tableofcontents[currentsection,currentsubsection]
%  \end{frame}
%}

\begin{document}

%Титульный слайд
\frame[plain]{\titlepage}

\begin{frame}
	\frametitle{Актуальность задачи}
	
	В настоящее время появляется всё больше интерактивных приложений в которых требуется визуализировать 3D-сцены. Часто они связаны с 3D-моделированием и визуализацией архитектуры. 
	
	Также отдельного внимания заслуживают VR-приложения.
\end{frame}

\begin{frame}
	\frametitle{Актуальность задачи}
	\begin{columns}[T]
    	\begin{column}[T]{5cm}
			Требования к VR-приложениям:
    		
		    \begin{enumerate}
		    	\item Визуализация на два экрана из разных позиций
		    	\item Разрешение экрана 1920x1080 и более
		    	\item Высокая частота кадров 90 и 120 кадров в секунду
		    \end{enumerate}
	    \end{column}
    	\begin{column}[T]{5cm}
			\includegraphics[width=\textwidth]{VR}
    	\end{column}
    \end{columns}
\end{frame}

\begin{frame}
	\frametitle{Постановка задачи}
	
	Необходимо, используя данные о~геометрии и~материалах сцены, перенести основную вычислительную сложность расчёта вторичного освещения со стадии визуализации на стадию предобработки и таким образом ускорить вычисление глобального освещения.

	\includegraphics[width=\textwidth]{task}
\end{frame}

\begin{frame}
	\frametitle{Алгоритм излучательности}
	\begin{columns}[T]
    	\begin{column}[T]{5cm}
			Алгоритм моделирует перенос энергии между площадками сцены.
			
			Форм-фактор для i-го и j-го треугольников --- число показывающее, какая часть энергии переходит с i-го треугольника на j-ый.
	    \end{column}
    	\begin{column}[T]{5cm}
			\includegraphics[width=\textwidth]{Radiosity}
    	\end{column}
    \end{columns}
\end{frame}

\begin{frame}
	\frametitle{Матрица форм-факторов}
	\begin{columns}[T]
    	\begin{column}[T]{5cm}
			\includegraphics[width=\textwidth]{FF_scheme}
    	\end{column}
    	\begin{column}[T]{5cm}
			Умножая световую энергию от первичного освещения на матрицу форм-факторов получаем освещение после первого отражения.
			
			Количество элементов в матрице достигает десятков миллионов.
	    \end{column}
    \end{columns}
\end{frame}

\begin{frame}
	\frametitle{Уменьшение количества умножений на матрицу}
	$F_C$ --- "цветная" матрица форм-факторов
	
	$F_C \cdot light$ --- свет, пришедший после первого отражения

	$F_C \cdot F_C \cdot light$ --- свет, пришедший после второго отражения
	
	$\cdots$
	
	$(F_C^3 + F_C^2 + F_C) \cdot light$ --- вторичное освещение после трех отражений
	
	$F_C^3 + F_C^2 + F_C$ --- зависит только от геометрии сцены и, следовательно, может быть посчитанно на этапе предрасчета
	
	\center
	\includegraphics[scale=0.4]{light_transport_1}
\end{frame}

\begin{frame}
	\frametitle{Уменьшение количества умножений на матрицу}
	$F_C$ --- "цветная" матрица форм-факторов
	
	$F_C \cdot light$ --- свет, пришедший после первого отражения

	$F_C \cdot F_C \cdot light$ --- свет, пришедший после второго отражения
	
	$\cdots$
	
	$(F_C^3 + F_C^2 + F_C) \cdot light$ --- вторичное освещение после трех отражений
	
	$F_C^3 + F_C^2 + F_C$ --- зависит только от геометрии сцены и, следовательно, может быть посчитанно на этапе предрасчета
	
	\center
	\includegraphics[scale=0.4]{light_transport_2}
\end{frame}

\begin{frame}
	\frametitle{Уменьшение количества умножений на матрицу}
	$F_C$ --- "цветная" матрица форм-факторов
	
	$F_C \cdot light$ --- свет, пришедший после первого отражения

	$F_C \cdot F_C \cdot light$ --- свет, пришедший после второго отражения
	
	$\cdots$
	
	$(F_C^3 + F_C^2 + F_C) \cdot light$ --- вторичное освещение после трех отражений
	
	$F_C^3 + F_C^2 + F_C$ --- зависит только от геометрии сцены и, следовательно, может быть посчитанно на этапе предрасчета
	
	\center
	\includegraphics[scale=0.4]{light_transport_3}
\end{frame}

\begin{frame}
	\frametitle{Уменьшение количества умножений на матрицу}
	$F_C$ --- "цветная" матрица форм-факторов
	
	$F_C \cdot light$ --- свет, пришедший после первого отражения

	$F_C \cdot F_C \cdot light$ --- свет, пришедший после второго отражения
	
	$\cdots$
	
	$(F_C^3 + F_C^2 + F_C) \cdot light$ --- вторичное освещение после трех отражений
	
	$F_C^3 + F_C^2 + F_C$ --- зависит только от геометрии сцены и, следовательно, может быть посчитанно на этапе предрасчета
	
	\center
	\includegraphics[scale=0.4]{light_transport_4}
\end{frame}

\begin{frame}
	\frametitle{Уменьшение количества умножений на матрицу}
	$F_C$ --- "цветная" матрица форм-факторов
	
	$F_C \cdot light$ --- свет, пришедший после первого отражения

	$F_C \cdot F_C \cdot light$ --- свет, пришедший после второго отражения
	
	$\cdots$
	
	$(F_C^3 + F_C^2 + F_C) \cdot light$ --- вторичное освещение после трех отражений
	
	$F_C^3 + F_C^2 + F_C$ --- зависит только от геометрии сцены и, следовательно, может быть посчитанно на этапе предрасчета
	
	\center
	\includegraphics[scale=0.4]{light_transport_5}
\end{frame}

\begin{frame}
	\frametitle{Уменьшение количества умножений на матрицу}
	$F_C$ --- "цветная" матрица форм-факторов
	
	$F_C \cdot light$ --- свет, пришедший после первого отражения

	$F_C \cdot F_C \cdot light$ --- свет, пришедший после второго отражения
	
	$\cdots$
	
	$(F_C^3 + F_C^2 + F_C) \cdot light$ --- вторичное освещение после трех отражений
	
	$F_C^3 + F_C^2 + F_C$ --- зависит только от геометрии сцены и, следовательно, может быть посчитанно на этапе предрасчета
	
	\center
	\includegraphics[scale=0.4]{light_transport_6}
\end{frame}

\begin{frame}
	\frametitle{Уменьшение количества умножений на матрицу}
	$F_C$ --- "цветная" матрица форм-факторов
	
	$F_C \cdot light$ --- свет, пришедший после первого отражения

	$F_C \cdot F_C \cdot light$ --- свет, пришедший после второго отражения
	
	$\cdots$
	
	$(F_C^3 + F_C^2 + F_C) \cdot light$ --- вторичное освещение после трех отражений
	
	$F_C^3 + F_C^2 + F_C$ --- зависит только от геометрии сцены и, следовательно, может быть посчитанно на этапе предрасчета
	
	\center
	\includegraphics[scale=0.4]{light_transport_7}
\end{frame}

\begin{frame}
	\frametitle{Переупорядочивание строк матрицы}
	\begin{columns}[T]
    	\begin{column}[T]{5cm}
			\includegraphics[width=\textwidth]{4}
    	\end{column}
    	\begin{column}[T]{5cm}
			\includegraphics[width=\textwidth]{6}
	    \end{column}
    \end{columns}
\end{frame}

\begin{frame}
	\frametitle{DXT-сжатие}
	\includegraphics[width=\textwidth]{31}
\end{frame}

\begin{frame}
	\frametitle{Результаты | Размеры файлов форм-факторов}
	\includegraphics[width=\textwidth]{23}
\end{frame}

\begin{frame}
	\frametitle{Результаты | Время выполнения алгоритма}
	\includegraphics[width=\textwidth]{24}
\end{frame}

\begin{frame}
	\frametitle{Результаты | Сравнение изображений}
	\center
	\includegraphics[width=0.8\textwidth]{26}
\end{frame}

\begin{frame}
	\frametitle{Результаты | Сравнение изображений}
	\center
	\includegraphics[width=0.8\textwidth]{28}
\end{frame}

\begin{frame}
	\frametitle{Практическая реализация}

	\begin{itemize}
		\item 2 программы: предподсчёт и визуализация
		\item 4800+ строк программного кода
		\item Использовались OpenGL и OpenCL, а так же библиотека SOIL
	\end{itemize}
\end{frame}

\begin{frame}
	\frametitle{Публикации}
	
	\begin{itemize}
		\item Автоматическое упрощение геометрии для расчёта вторичной освещенности методом излучательности. 
		\textit{Сборник трудов Графикон 2016}
		\item Оптимизация матрицы форм-факторов для эффективной реализации излучательности на графические процессоры в приложениях реального времени.
		\textit{Сборник тезисов XXIV конференции Ломоносов}
		\item Accelerating radiosity on GPUs. 
		\textit{WSCG2017 25th International Conference in Central Europe on Computer Graphics, Visualization and Computer Vision 2017}
	\end{itemize}
\end{frame}

\end{document}
